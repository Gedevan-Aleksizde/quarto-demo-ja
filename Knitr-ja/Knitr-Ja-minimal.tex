% Options for packages loaded elsewhere
\PassOptionsToPackage{unicode}{hyperref}
\PassOptionsToPackage{hyphens}{url}
\PassOptionsToPackage{dvipsnames,svgnames,x11names}{xcolor}
%
\documentclass[
  letterpaper,
  DIV=11,
  pandoc,
  ja=standard,
  jafont=noto-otf]{bxjsarticle}
\usepackage{amsmath,amssymb}
\usepackage{lmodern}
\usepackage{iftex}
\ifPDFTeX
  \usepackage[T1]{fontenc}
  \usepackage[utf8]{inputenc}
  \usepackage{textcomp} % provide euro and other symbols
\else % if luatex or xetex
  \usepackage{unicode-math}
  \defaultfontfeatures{Scale=MatchLowercase}
  \defaultfontfeatures[\rmfamily]{Ligatures=TeX,Scale=1}
  \setmainfont[]{Deja Vu Serif}
  \setsansfont[]{Deja Vu Sans}
  \setmonofont[]{Iosevka Term Extended}
\fi
% Use upquote if available, for straight quotes in verbatim environments
\IfFileExists{upquote.sty}{\usepackage{upquote}}{}
\IfFileExists{microtype.sty}{% use microtype if available
  \usepackage[]{microtype}
  \UseMicrotypeSet[protrusion]{basicmath} % disable protrusion for tt fonts
}{}
\makeatletter
\@ifundefined{KOMAClassName}{% if non-KOMA class
  \IfFileExists{parskip.sty}{%
    \usepackage{parskip}
  }{% else
    \setlength{\parindent}{0pt}
    \setlength{\parskip}{6pt plus 2pt minus 1pt}}
}{% if KOMA class
  \KOMAoptions{parskip=half}}
\makeatother
\usepackage{xcolor}
\IfFileExists{xurl.sty}{\usepackage{xurl}}{} % add URL line breaks if available
\IfFileExists{bookmark.sty}{\usepackage{bookmark}}{\usepackage{hyperref}}
\hypersetup{
  pdftitle={比較的ミニマルな Quarto の日本語組版 PDF 文書設定},
  colorlinks=true,
  linkcolor={blue},
  filecolor={Maroon},
  citecolor={Blue},
  urlcolor={Blue},
  pdfcreator={LaTeX via pandoc}}
\urlstyle{same} % disable monospaced font for URLs
\usepackage{color}
\usepackage{fancyvrb}
\newcommand{\VerbBar}{|}
\newcommand{\VERB}{\Verb[commandchars=\\\{\}]}
\DefineVerbatimEnvironment{Highlighting}{Verbatim}{commandchars=\\\{\}}
% Add ',fontsize=\small' for more characters per line
\usepackage{framed}
\definecolor{shadecolor}{RGB}{241,243,245}
\newenvironment{Shaded}{\begin{snugshade}}{\end{snugshade}}
\newcommand{\AlertTok}[1]{\textcolor[rgb]{0.68,0.00,0.00}{#1}}
\newcommand{\AnnotationTok}[1]{\textcolor[rgb]{0.37,0.37,0.37}{#1}}
\newcommand{\AttributeTok}[1]{\textcolor[rgb]{0.00,0.48,0.65}{#1}}
\newcommand{\BaseNTok}[1]{\textcolor[rgb]{0.68,0.00,0.00}{#1}}
\newcommand{\BuiltInTok}[1]{\textcolor[rgb]{0.00,0.48,0.65}{#1}}
\newcommand{\CharTok}[1]{\textcolor[rgb]{0.13,0.47,0.30}{#1}}
\newcommand{\CommentTok}[1]{\textcolor[rgb]{0.37,0.37,0.37}{#1}}
\newcommand{\CommentVarTok}[1]{\textcolor[rgb]{0.37,0.37,0.37}{\textit{#1}}}
\newcommand{\ConstantTok}[1]{\textcolor[rgb]{0.56,0.35,0.01}{#1}}
\newcommand{\ControlFlowTok}[1]{\textcolor[rgb]{0.00,0.48,0.65}{#1}}
\newcommand{\DataTypeTok}[1]{\textcolor[rgb]{0.68,0.00,0.00}{#1}}
\newcommand{\DecValTok}[1]{\textcolor[rgb]{0.68,0.00,0.00}{#1}}
\newcommand{\DocumentationTok}[1]{\textcolor[rgb]{0.37,0.37,0.37}{\textit{#1}}}
\newcommand{\ErrorTok}[1]{\textcolor[rgb]{0.68,0.00,0.00}{#1}}
\newcommand{\ExtensionTok}[1]{\textcolor[rgb]{0.00,0.48,0.65}{#1}}
\newcommand{\FloatTok}[1]{\textcolor[rgb]{0.68,0.00,0.00}{#1}}
\newcommand{\FunctionTok}[1]{\textcolor[rgb]{0.28,0.35,0.67}{#1}}
\newcommand{\ImportTok}[1]{\textcolor[rgb]{0.00,0.48,0.65}{#1}}
\newcommand{\InformationTok}[1]{\textcolor[rgb]{0.37,0.37,0.37}{#1}}
\newcommand{\KeywordTok}[1]{\textcolor[rgb]{0.00,0.48,0.65}{#1}}
\newcommand{\NormalTok}[1]{\textcolor[rgb]{0.00,0.48,0.65}{#1}}
\newcommand{\OperatorTok}[1]{\textcolor[rgb]{0.37,0.37,0.37}{#1}}
\newcommand{\OtherTok}[1]{\textcolor[rgb]{0.00,0.48,0.65}{#1}}
\newcommand{\PreprocessorTok}[1]{\textcolor[rgb]{0.68,0.00,0.00}{#1}}
\newcommand{\RegionMarkerTok}[1]{\textcolor[rgb]{0.00,0.48,0.65}{#1}}
\newcommand{\SpecialCharTok}[1]{\textcolor[rgb]{0.37,0.37,0.37}{#1}}
\newcommand{\SpecialStringTok}[1]{\textcolor[rgb]{0.13,0.47,0.30}{#1}}
\newcommand{\StringTok}[1]{\textcolor[rgb]{0.13,0.47,0.30}{#1}}
\newcommand{\VariableTok}[1]{\textcolor[rgb]{0.07,0.07,0.07}{#1}}
\newcommand{\VerbatimStringTok}[1]{\textcolor[rgb]{0.13,0.47,0.30}{#1}}
\newcommand{\WarningTok}[1]{\textcolor[rgb]{0.37,0.37,0.37}{\textit{#1}}}
\usepackage{longtable,booktabs,array}
\usepackage{calc} % for calculating minipage widths
% Correct order of tables after \paragraph or \subparagraph
\usepackage{etoolbox}
\makeatletter
\patchcmd\longtable{\par}{\if@noskipsec\mbox{}\fi\par}{}{}
\makeatother
% Allow footnotes in longtable head/foot
\IfFileExists{footnotehyper.sty}{\usepackage{footnotehyper}}{\usepackage{footnote}}
\makesavenoteenv{longtable}
\usepackage{graphicx}
\makeatletter
\def\maxwidth{\ifdim\Gin@nat@width>\linewidth\linewidth\else\Gin@nat@width\fi}
\def\maxheight{\ifdim\Gin@nat@height>\textheight\textheight\else\Gin@nat@height\fi}
\makeatother
% Scale images if necessary, so that they will not overflow the page
% margins by default, and it is still possible to overwrite the defaults
% using explicit options in \includegraphics[width, height, ...]{}
\setkeys{Gin}{width=\maxwidth,height=\maxheight,keepaspectratio}
% Set default figure placement to htbp
\makeatletter
\def\fps@figure{htbp}
\makeatother
\setlength{\emergencystretch}{3em} % prevent overfull lines
\providecommand{\tightlist}{%
  \setlength{\itemsep}{0pt}\setlength{\parskip}{0pt}}
\setcounter{secnumdepth}{-\maxdimen} % remove section numbering
\makeatletter
\makeatother
\makeatletter
\@ifpackageloaded{caption}{}{\usepackage{caption}}
\AtBeginDocument{%
\renewcommand*\figurename{図}
\renewcommand*\tablename{表}
}
\AtBeginDocument{%
\renewcommand*\listfigurename{List of Figures}
\renewcommand*\listtablename{List of Tables}
}
\@ifpackageloaded{float}{}{\usepackage{float}}
\floatstyle{ruled}
\@ifundefined{c@chapter}{\newfloat{codelisting}{h}{lop}}{\newfloat{codelisting}{h}{lop}[chapter]}
\floatname{codelisting}{Listing}
\newcommand*\listoflistings{\listof{codelisting}{List of Listings}}
\makeatother
\makeatletter
\@ifpackageloaded{caption}{}{\usepackage{caption}}
\@ifpackageloaded{subfig}{}{\usepackage{subfig}}
\makeatother
\ifLuaTeX
  \usepackage{selnolig}  % disable illegal ligatures
\fi
\usepackage[style=authoryear]{biblatex}
\addbibresource{../quarto.bib}

\title{比較的ミニマルな Quarto の日本語組版 PDF 文書設定}
\author{}
\date{}

\begin{document}
\maketitle

\hypertarget{ux57faux672cux8a2dux5b9a}{%
\section{基本設定}\label{ux57faux672cux8a2dux5b9a}}

\begin{itemize}
\tightlist
\item
  LuaLaTeX または XeLaTeX でのコンパイルを想定している.

  \begin{itemize}
  \tightlist
  \item
    既定では LuaLaTeX を使用している. \texttt{pdf-engine:\ xelatex} で
    XeLaTeX を使用可能. R Markdown の \texttt{latex\_engine}
    に対応してるが, Quarto は Pandoc の同名のオプションに直接渡すようだ.
  \end{itemize}
\item
  \texttt{jafont=...} でフォントプリセットを設定可能. 指定可能な名称は
  LuaLaTeX/XeLaTeX それぞれ luatex-japreset, bxjscls (zxjafont)
  のドキュメント参照. 今回は比較的環境依存しない \texttt{haranoaji}
  (原ノ味) を採用.
\item
  \texttt{mainfont}/\texttt{sansfont}/\texttt{monofont:}
  はそれぞれメインフォント(通常は明朝体), サンセリフフォント
  (通常は見出しや太字で使用するゴシック体), 等幅フォントの指定.
  それぞれプリセットより優先される.

  \begin{itemize}
  \tightlist
  \item
    これはあくまでオプションの紹介として書いただけ.
    \textbf{多くの環境ではこの設定のままだとエラーが出るか文字化けするだろう}.
    良くても別の (見慣れない) フォントにフォールバックする可能性が高い.
    変更するか消してほしい.
  \item
    \textbf{rmdja}
    パッケージでは欧文と和文でさらに個別指定できるようになっているが,
    現状はそこまで細かい設定はできないし,
    ほとんどの人は気にしないだろう. むしろ煩雑にさえ感じるかもしれない.
  \end{itemize}
\end{itemize}

\hypertarget{markdown}{%
\section{Markdown}\label{markdown}}

\begin{enumerate}
\def\labelenumi{\arabic{enumi}.}
\tightlist
\item
  番号付きの
\item
  箇条書き

  \begin{enumerate}
  \def\labelenumii{\arabic{enumii}.}
  \tightlist
  \item
    ネストも
  \item
    できる
  \end{enumerate}
\end{enumerate}

\hypertarget{ux6570ux5f0fux306eux8868ux793a}{%
\subsection{数式の表示}\label{ux6570ux5f0fux306eux8868ux793a}}

ブラック=ショールズ方程式 (式~\ref{eq-black-scholes})

\begin{equation}\protect\hypertarget{eq-black-scholes}{}{
\frac{\partial \mathrm C}{ \partial \mathrm t } + \frac{1}{2}\sigma^{2} \mathrm S^{2}
\frac{\partial^{2} \mathrm C}{\partial \mathrm C^2}
  + \mathrm r \mathrm S \frac{\partial \mathrm C}{\partial \mathrm S}\ =
  \mathrm r \mathrm C 
}\label{eq-black-scholes}\end{equation}

HTML と PDF 双方で相互参照を使用したい場合, LaTeX の
\texttt{\textbackslash{}label()} ではなく Quarto の構文を使用する. KaTeX
も使えるが PDF と互換性があるとは限らない?

\hypertarget{ux30b3ux30fcux30c9ux306eux57cbux3081ux8fbcux307f}{%
\subsection{コードの埋め込み}\label{ux30b3ux30fcux30c9ux306eux57cbux3081ux8fbcux307f}}

図~\ref{fig-plot1-1}, 図~\ref{fig-plot1-2} を見よ.

\begin{Shaded}
\begin{Highlighting}[]
\FunctionTok{library}\NormalTok{(ggplot2)}
\FunctionTok{ggplot}\NormalTok{(airquality, }\FunctionTok{aes}\NormalTok{(Temp, Ozone)) }\SpecialCharTok{+} 
        \FunctionTok{geom\_point}\NormalTok{() }\SpecialCharTok{+} 
        \FunctionTok{geom\_smooth}\NormalTok{(}\AttributeTok{method =} \StringTok{"loess"}\NormalTok{, }\AttributeTok{se =}\NormalTok{ F, }\AttributeTok{formula =}\NormalTok{ y }\SpecialCharTok{\textasciitilde{}}\NormalTok{ x)}

\FunctionTok{ggplot}\NormalTok{(mtcars, }\FunctionTok{aes}\NormalTok{(}\AttributeTok{x =} \FunctionTok{factor}\NormalTok{(cyl), }\AttributeTok{y =}\NormalTok{ mpg)) }\SpecialCharTok{+} \FunctionTok{geom\_boxplot}\NormalTok{()}
\end{Highlighting}
\end{Shaded}

\begin{figure}

\subfloat[Air Quality]{\label{fig-plot1-1}%
\begin{minipage}[t]{0.50\linewidth}
\raisebox{-\height}{

{\centering 

\includegraphics{Knitr-Ja-minimal_files/figure-pdf/fig-plot1-1.pdf}

}

}\end{minipage}%
}
%
\subfloat[箱ひげ図]{\label{fig-plot1-2}%
\begin{minipage}[t]{0.50\linewidth}
\raisebox{-\height}{

{\centering 

\includegraphics{Knitr-Ja-minimal_files/figure-pdf/fig-plot1-2.pdf}

}

}\end{minipage}%
}

\caption{\label{fig-plot1}複数の図}

\end{figure}

次に 表~\ref{tbl-tables-cars}, 表~\ref{tbl-tables-pressure} を見よ.

\begin{table}

\subfloat[Cars ]{\label{tbl-tables-cars}%
\begin{minipage}[t]{0.50\linewidth}

{\centering 

\begin{tabular}[t]{rr}
\toprule
speed & dist\\
\midrule
4 & 2\\
4 & 10\\
7 & 4\\
7 & 22\\
8 & 16\\
9 & 10\\
\bottomrule
\end{tabular}

}

\end{minipage}%
}
%
\subfloat[Pressure ]{\label{tbl-tables-pressure}%
\begin{minipage}[t]{0.50\linewidth}

{\centering 

\begin{tabular}[t]{rr}
\toprule
temperature & pressure\\
\midrule
0 & 0.0002\\
20 & 0.0012\\
40 & 0.0060\\
60 & 0.0300\\
80 & 0.0900\\
100 & 0.2700\\
\bottomrule
\end{tabular}

}

\end{minipage}%
}

\caption{\label{tbl-tables}複数の表}

\end{table}

\hypertarget{ux6587ux732eux5f15ux7528}{%
\section{文献引用}\label{ux6587ux732eux5f15ux7528}}

\autocite{R-quarto}, \textcite{R-rmdja}

\printbibliography[title=参考文献]

\end{document}
