% Options for packages loaded elsewhere
\PassOptionsToPackage{unicode}{hyperref}
\PassOptionsToPackage{hyphens}{url}
\PassOptionsToPackage{dvipsnames,svgnames,x11names}{xcolor}
%
\documentclass[
  ignorenonframetext,
]{beamer}
\usepackage{pgfpages}
\setbeamertemplate{caption}[numbered]
\setbeamertemplate{caption label separator}{: }
\setbeamercolor{caption name}{fg=normal text.fg}
\beamertemplatenavigationsymbolsempty
% Prevent slide breaks in the middle of a paragraph
\widowpenalties 1 10000
\raggedbottom
\setbeamertemplate{part page}{
  \centering
  \begin{beamercolorbox}[sep=16pt,center]{part title}
    \usebeamerfont{part title}\insertpart\par
  \end{beamercolorbox}
}
\setbeamertemplate{section page}{
  \centering
  \begin{beamercolorbox}[sep=12pt,center]{part title}
    \usebeamerfont{section title}\insertsection\par
  \end{beamercolorbox}
}
\setbeamertemplate{subsection page}{
  \centering
  \begin{beamercolorbox}[sep=8pt,center]{part title}
    \usebeamerfont{subsection title}\insertsubsection\par
  \end{beamercolorbox}
}
\AtBeginPart{
  \frame{\partpage}
}
\AtBeginSection{
  \ifbibliography
  \else
    \frame{\sectionpage}
  \fi
}
\AtBeginSubsection{
  \frame{\subsectionpage}
}
\usepackage{amsmath,amssymb}
\usepackage{lmodern}
\usepackage{iftex}
\ifPDFTeX
  \usepackage[T1]{fontenc}
  \usepackage[utf8]{inputenc}
  \usepackage{textcomp} % provide euro and other symbols
\else % if luatex or xetex
  \usepackage{unicode-math}
  \defaultfontfeatures{Scale=MatchLowercase}
  \defaultfontfeatures[\rmfamily]{Ligatures=TeX,Scale=1}
  \setmonofont[]{Iosevka Term Extended}
\fi
\usetheme[block=fill,background=dark]{metropolis}
% Use upquote if available, for straight quotes in verbatim environments
\IfFileExists{upquote.sty}{\usepackage{upquote}}{}
\IfFileExists{microtype.sty}{% use microtype if available
  \usepackage[]{microtype}
  \UseMicrotypeSet[protrusion]{basicmath} % disable protrusion for tt fonts
}{}
\makeatletter
\@ifundefined{KOMAClassName}{% if non-KOMA class
  \IfFileExists{parskip.sty}{%
    \usepackage{parskip}
  }{% else
    \setlength{\parindent}{0pt}
    \setlength{\parskip}{6pt plus 2pt minus 1pt}}
}{% if KOMA class
  \KOMAoptions{parskip=half}}
\makeatother
\usepackage{xcolor}
\IfFileExists{xurl.sty}{\usepackage{xurl}}{} % add URL line breaks if available
\IfFileExists{bookmark.sty}{\usepackage{bookmark}}{\usepackage{hyperref}}
\hypersetup{
  pdftitle={日本語対応スライドの例 (Quarto + knitr エンジン)},
  pdfauthor={ill-identified},
  colorlinks=true,
  linkcolor={blue},
  filecolor={Maroon},
  citecolor={Blue},
  urlcolor={Blue},
  pdfcreator={LaTeX via pandoc}}
\urlstyle{same} % disable monospaced font for URLs
\newif\ifbibliography
\usepackage{color}
\usepackage{fancyvrb}
\newcommand{\VerbBar}{|}
\newcommand{\VERB}{\Verb[commandchars=\\\{\}]}
\DefineVerbatimEnvironment{Highlighting}{Verbatim}{commandchars=\\\{\}}
% Add ',fontsize=\small' for more characters per line
\usepackage{framed}
\definecolor{shadecolor}{RGB}{241,243,245}
\newenvironment{Shaded}{\begin{snugshade}}{\end{snugshade}}
\newcommand{\AlertTok}[1]{\textcolor[rgb]{0.68,0.00,0.00}{#1}}
\newcommand{\AnnotationTok}[1]{\textcolor[rgb]{0.37,0.37,0.37}{#1}}
\newcommand{\AttributeTok}[1]{\textcolor[rgb]{0.00,0.48,0.65}{#1}}
\newcommand{\BaseNTok}[1]{\textcolor[rgb]{0.68,0.00,0.00}{#1}}
\newcommand{\BuiltInTok}[1]{\textcolor[rgb]{0.00,0.48,0.65}{#1}}
\newcommand{\CharTok}[1]{\textcolor[rgb]{0.13,0.47,0.30}{#1}}
\newcommand{\CommentTok}[1]{\textcolor[rgb]{0.37,0.37,0.37}{#1}}
\newcommand{\CommentVarTok}[1]{\textcolor[rgb]{0.37,0.37,0.37}{\textit{#1}}}
\newcommand{\ConstantTok}[1]{\textcolor[rgb]{0.56,0.35,0.01}{#1}}
\newcommand{\ControlFlowTok}[1]{\textcolor[rgb]{0.00,0.48,0.65}{#1}}
\newcommand{\DataTypeTok}[1]{\textcolor[rgb]{0.68,0.00,0.00}{#1}}
\newcommand{\DecValTok}[1]{\textcolor[rgb]{0.68,0.00,0.00}{#1}}
\newcommand{\DocumentationTok}[1]{\textcolor[rgb]{0.37,0.37,0.37}{\textit{#1}}}
\newcommand{\ErrorTok}[1]{\textcolor[rgb]{0.68,0.00,0.00}{#1}}
\newcommand{\ExtensionTok}[1]{\textcolor[rgb]{0.00,0.48,0.65}{#1}}
\newcommand{\FloatTok}[1]{\textcolor[rgb]{0.68,0.00,0.00}{#1}}
\newcommand{\FunctionTok}[1]{\textcolor[rgb]{0.28,0.35,0.67}{#1}}
\newcommand{\ImportTok}[1]{\textcolor[rgb]{0.00,0.48,0.65}{#1}}
\newcommand{\InformationTok}[1]{\textcolor[rgb]{0.37,0.37,0.37}{#1}}
\newcommand{\KeywordTok}[1]{\textcolor[rgb]{0.00,0.48,0.65}{#1}}
\newcommand{\NormalTok}[1]{\textcolor[rgb]{0.00,0.48,0.65}{#1}}
\newcommand{\OperatorTok}[1]{\textcolor[rgb]{0.37,0.37,0.37}{#1}}
\newcommand{\OtherTok}[1]{\textcolor[rgb]{0.00,0.48,0.65}{#1}}
\newcommand{\PreprocessorTok}[1]{\textcolor[rgb]{0.68,0.00,0.00}{#1}}
\newcommand{\RegionMarkerTok}[1]{\textcolor[rgb]{0.00,0.48,0.65}{#1}}
\newcommand{\SpecialCharTok}[1]{\textcolor[rgb]{0.37,0.37,0.37}{#1}}
\newcommand{\SpecialStringTok}[1]{\textcolor[rgb]{0.13,0.47,0.30}{#1}}
\newcommand{\StringTok}[1]{\textcolor[rgb]{0.13,0.47,0.30}{#1}}
\newcommand{\VariableTok}[1]{\textcolor[rgb]{0.07,0.07,0.07}{#1}}
\newcommand{\VerbatimStringTok}[1]{\textcolor[rgb]{0.13,0.47,0.30}{#1}}
\newcommand{\WarningTok}[1]{\textcolor[rgb]{0.37,0.37,0.37}{\textit{#1}}}
\usepackage{longtable,booktabs,array}
\usepackage{calc} % for calculating minipage widths
\usepackage{caption}
% Make caption package work with longtable
\makeatletter
\def\fnum@table{\tablename~\thetable}
\makeatother
\usepackage{graphicx}
\makeatletter
\def\maxwidth{\ifdim\Gin@nat@width>\linewidth\linewidth\else\Gin@nat@width\fi}
\def\maxheight{\ifdim\Gin@nat@height>\textheight\textheight\else\Gin@nat@height\fi}
\makeatother
% Scale images if necessary, so that they will not overflow the page
% margins by default, and it is still possible to overwrite the defaults
% using explicit options in \includegraphics[width, height, ...]{}
\setkeys{Gin}{width=\maxwidth,height=\maxheight,keepaspectratio}
% Set default figure placement to htbp
\makeatletter
\def\fps@figure{htbp}
\makeatother
\setlength{\emergencystretch}{3em} % prevent overfull lines
\providecommand{\tightlist}{%
  \setlength{\itemsep}{0pt}\setlength{\parskip}{0pt}}
\setcounter{secnumdepth}{-\maxdimen} % remove section numbering
\newlength{\cslhangindent}
\setlength{\cslhangindent}{1.5em}
\newlength{\csllabelwidth}
\setlength{\csllabelwidth}{3em}
\newlength{\cslentryspacingunit} % times entry-spacing
\setlength{\cslentryspacingunit}{\parskip}
\newenvironment{CSLReferences}[2] % #1 hanging-ident, #2 entry spacing
 {% don't indent paragraphs
  \setlength{\parindent}{0pt}
  % turn on hanging indent if param 1 is 1
  \ifodd #1
  \let\oldpar\par
  \def\par{\hangindent=\cslhangindent\oldpar}
  \fi
  % set entry spacing
  \setlength{\parskip}{#2\cslentryspacingunit}
 }%
 {}
\usepackage{calc}
\newcommand{\CSLBlock}[1]{#1\hfill\break}
\newcommand{\CSLLeftMargin}[1]{\parbox[t]{\csllabelwidth}{#1}}
\newcommand{\CSLRightInline}[1]{\parbox[t]{\linewidth - \csllabelwidth}{#1}\break}
\newcommand{\CSLIndent}[1]{\hspace{\cslhangindent}#1}
\usepackage[haranoaji]{zxjafont}
\setbeamercolor{frametitle}{bg=Gray,fg=White}
\makeatletter
\makeatother
\makeatletter
\@ifpackageloaded{caption}{}{\usepackage{caption}}
\AtBeginDocument{%
\renewcommand*\figurename{図}
\renewcommand*\tablename{表}
}
\AtBeginDocument{%
\renewcommand*\listfigurename{List of Figures}
\renewcommand*\listtablename{List of Tables}
}
\@ifpackageloaded{float}{}{\usepackage{float}}
\floatstyle{ruled}
\@ifundefined{c@chapter}{\newfloat{codelisting}{h}{lop}}{\newfloat{codelisting}{h}{lop}[chapter]}
\floatname{codelisting}{Listing}
\newcommand*\listoflistings{\listof{codelisting}{List of Listings}}
\makeatother
\makeatletter
\@ifpackageloaded{caption}{}{\usepackage{caption}}
\@ifpackageloaded{subfig}{}{\usepackage{subfig}}
\makeatother
\ifLuaTeX
  \usepackage{selnolig}  % disable illegal ligatures
\fi

\title{日本語対応スライドの例 (Quarto + knitr エンジン)}
\author{ill-identified}
\date{2021/10/2}

\begin{document}
\frame{\titlepage}

\begin{frame}{}
\protect\hypertarget{section}{}
\begin{itemize}
\tightlist
\item
  Quarto は最近公開されたばかりで開発中
\item
  \textbf{ここの記述もすぐ時代遅れになる可能性がある}ことに注意
\end{itemize}
\end{frame}

\begin{frame}[fragile]{YAML メタデータの解説}
\protect\hypertarget{yaml-ux30e1ux30bfux30c7ux30fcux30bfux306eux89e3ux8aac}{}
\begin{itemize}
\tightlist
\item
  Beamer について

  \begin{itemize}
  \tightlist
  \item
    metropolis テーマ使用
  \item
    metropolis は XeLaTeX 使用を想定している
  \item
    しかし XeLaTeX では FontAwesome がうまく認識されない? ので callout
    ブロックは使用できない.
  \item
    フォントプリセット指定や細かい設定変更は現状 LaTeX
    コマンドで書くしかない
  \end{itemize}
\item
  revealjs について

  \begin{itemize}
  \tightlist
  \item
    全体的に表示がうまくいってない
  \item
    デフォルトのデザインもあまりよくない
  \end{itemize}
\item
  スライドは余白が貴重なので, コードを表示しないデフォルト設定に
\item
  \texttt{dev} は効果があるのかよくわからん
\end{itemize}
\end{frame}

\begin{frame}{Markdown の例}
\protect\hypertarget{markdown-ux306eux4f8b}{}
\begin{itemize}
\tightlist
\item
  箇条書き
\end{itemize}

\begin{enumerate}
\tightlist
\item
  aaa
\item
  bbb
\item
  ccc
\end{enumerate}
\end{frame}

\begin{frame}[fragile]{ブロック構文}
\protect\hypertarget{ux30d6ux30edux30c3ux30afux69cbux6587}{}
\begin{itemize}
\tightlist
\item
  以下は Beamer のブロックの出力例
\item
  Beamer 以外では機能しない可能性
\end{itemize}

\begin{alertblock}{ブロック}
これは \texttt{block} 環境

\end{alertblock}

\begin{alertblock}{警告ブロック}
これは \texttt{alertblock} 環境

\end{alertblock}

\begin{exampleblock}{用例ブロック}
これは \texttt{exampleblock} 環境

\end{exampleblock}
\end{frame}

\begin{frame}{数式表示}
\protect\hypertarget{ux6570ux5f0fux8868ux793a}{}
\begin{itemize}
\tightlist
\item
  ブラック=ショールズ方程式 (式~\ref{eq-black-scholes})
\end{itemize}

\begin{equation}\protect\hypertarget{eq-black-scholes}{}{
\frac{\partial \mathrm C}{ \partial \mathrm t } + \frac{1}{2}\sigma^{2} \mathrm S^{2}
\frac{\partial^{2} \mathrm C}{\partial \mathrm C^2}
  + \mathrm r \mathrm S \frac{\partial \mathrm C}{\partial \mathrm S}\ =
  \mathrm r \mathrm C 
}\label{eq-black-scholes}\end{equation}
\end{frame}

\begin{frame}[fragile]{コード}
\protect\hypertarget{ux30b3ux30fcux30c9}{}
\begin{itemize}
\tightlist
\item
  コードのみで実行しない
\end{itemize}

\begin{Shaded}
\begin{Highlighting}[]
\FunctionTok{require}\NormalTok{(ggplot2)}

\FunctionTok{ggplot}\NormalTok{(mtcars, }\FunctionTok{aes}\NormalTok{(wt, mpg, }\AttributeTok{color =} \FunctionTok{factor}\NormalTok{(gear))) }\SpecialCharTok{+}
  \FunctionTok{geom\_point}\NormalTok{() }\SpecialCharTok{+}
  \FunctionTok{stat\_smooth}\NormalTok{(}\AttributeTok{method =} \StringTok{"lm"}\NormalTok{) }\SpecialCharTok{+}
  \FunctionTok{facet\_wrap}\NormalTok{(}\SpecialCharTok{\textasciitilde{}}\NormalTok{gear)}
\end{Highlighting}
\end{Shaded}
\end{frame}

\begin{frame}{グラフ生成コード埋め込みの例}
\protect\hypertarget{ux30b0ux30e9ux30d5ux751fux6210ux30b3ux30fcux30c9ux57cbux3081ux8fbcux307fux306eux4f8b}{}
\begin{figure}

{\centering \includegraphics[width=\textwidth,height=0.8\textheight]{Knitr-slide_files/figure-beamer/fig-plot-1.pdf}

}

\caption{\label{fig-plot}Rで描画したグラフ}

\end{figure}
\end{frame}

\begin{frame}{表}
\protect\hypertarget{ux8868}{}
\begin{itemize}
\tightlist
\item
  表~\ref{tbl-tables-cars}, 表~\ref{tbl-tables-pressure} を並べて表示
\end{itemize}

\begin{table}

\subfloat[Cars データセット ]{\label{tbl-tables-cars}%
\begin{minipage}[t]{0.50\linewidth}

{\centering 

\begin{tabular}[t]{rr}
\toprule
speed & dist\\
\midrule
4 & 2\\
4 & 10\\
7 & 4\\
7 & 22\\
8 & 16\\
9 & 10\\
\bottomrule
\end{tabular}

}

\end{minipage}%
}
%
\subfloat[Pressure データセット ]{\label{tbl-tables-pressure}%
\begin{minipage}[t]{0.50\linewidth}

{\centering 

\begin{tabular}[t]{rr}
\toprule
temperature & pressure\\
\midrule
0 & 0.0002\\
20 & 0.0012\\
40 & 0.0060\\
60 & 0.0300\\
80 & 0.0900\\
100 & 0.2700\\
\bottomrule
\end{tabular}

}

\end{minipage}%
}

\caption{\label{tbl-tables}複数の表}

\end{table}
\end{frame}

\begin{frame}{文献引用}
\protect\hypertarget{ux6587ux732eux5f15ux7528}{}
Allaire (2021), 片桐 (2021)

\hypertarget{refs}{}
\begin{CSLReferences}{1}{0}
\leavevmode\vadjust pre{\hypertarget{ref-R-quarto}{}}%
Allaire, JJ. 2021. \emph{Quarto: R Interface to 'Quarto' Markdown
Publishing System}. \url{https://CRAN.R-project.org/package=quarto}.

\leavevmode\vadjust pre{\hypertarget{ref-R-rmdja}{}}%
片桐智志. 2021. {``Rmdja: 日本語用 r Markdown テンプレート.''}
\url{https://github.com/Gedevan-Aleksizde/rmdja}.

\end{CSLReferences}
\end{frame}

\end{document}
