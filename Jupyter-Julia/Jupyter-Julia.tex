% Options for packages loaded elsewhere
\PassOptionsToPackage{unicode}{hyperref}
\PassOptionsToPackage{hyphens}{url}
\PassOptionsToPackage{dvipsnames,svgnames,x11names}{xcolor}
%
\documentclass[
  letterpaper,
  DIV=11,
  pandoc,
  ja=standard,
  jafont=haranoaji]{bxjsarticle}
\usepackage{amsmath,amssymb}
\usepackage{lmodern}
\usepackage{iftex}
\ifPDFTeX
  \usepackage[T1]{fontenc}
  \usepackage[utf8]{inputenc}
  \usepackage{textcomp} % provide euro and other symbols
\else % if luatex or xetex
  \usepackage{unicode-math}
  \defaultfontfeatures{Scale=MatchLowercase}
  \defaultfontfeatures[\rmfamily]{Ligatures=TeX,Scale=1}
\fi
% Use upquote if available, for straight quotes in verbatim environments
\IfFileExists{upquote.sty}{\usepackage{upquote}}{}
\IfFileExists{microtype.sty}{% use microtype if available
  \usepackage[]{microtype}
  \UseMicrotypeSet[protrusion]{basicmath} % disable protrusion for tt fonts
}{}
\makeatletter
\@ifundefined{KOMAClassName}{% if non-KOMA class
  \IfFileExists{parskip.sty}{%
    \usepackage{parskip}
  }{% else
    \setlength{\parindent}{0pt}
    \setlength{\parskip}{6pt plus 2pt minus 1pt}}
}{% if KOMA class
  \KOMAoptions{parskip=half}}
\makeatother
\usepackage{xcolor}
\IfFileExists{xurl.sty}{\usepackage{xurl}}{} % add URL line breaks if available
\IfFileExists{bookmark.sty}{\usepackage{bookmark}}{\usepackage{hyperref}}
\hypersetup{
  pdftitle={Quarto 使用例 (Jupyter + Julia 1.6.3)},
  pdfauthor={ill-identified},
  colorlinks=true,
  linkcolor={blue},
  filecolor={Maroon},
  citecolor={Blue},
  urlcolor={Blue},
  pdfcreator={LaTeX via pandoc}}
\urlstyle{same} % disable monospaced font for URLs
\usepackage{color}
\usepackage{fancyvrb}
\newcommand{\VerbBar}{|}
\newcommand{\VERB}{\Verb[commandchars=\\\{\}]}
\DefineVerbatimEnvironment{Highlighting}{Verbatim}{commandchars=\\\{\}}
% Add ',fontsize=\small' for more characters per line
\usepackage{framed}
\definecolor{shadecolor}{RGB}{241,243,245}
\newenvironment{Shaded}{\begin{snugshade}}{\end{snugshade}}
\newcommand{\AlertTok}[1]{\textcolor[rgb]{0.68,0.00,0.00}{#1}}
\newcommand{\AnnotationTok}[1]{\textcolor[rgb]{0.37,0.37,0.37}{#1}}
\newcommand{\AttributeTok}[1]{\textcolor[rgb]{0.00,0.48,0.65}{#1}}
\newcommand{\BaseNTok}[1]{\textcolor[rgb]{0.68,0.00,0.00}{#1}}
\newcommand{\BuiltInTok}[1]{\textcolor[rgb]{0.00,0.48,0.65}{#1}}
\newcommand{\CharTok}[1]{\textcolor[rgb]{0.13,0.47,0.30}{#1}}
\newcommand{\CommentTok}[1]{\textcolor[rgb]{0.37,0.37,0.37}{#1}}
\newcommand{\CommentVarTok}[1]{\textcolor[rgb]{0.37,0.37,0.37}{\textit{#1}}}
\newcommand{\ConstantTok}[1]{\textcolor[rgb]{0.56,0.35,0.01}{#1}}
\newcommand{\ControlFlowTok}[1]{\textcolor[rgb]{0.00,0.48,0.65}{#1}}
\newcommand{\DataTypeTok}[1]{\textcolor[rgb]{0.68,0.00,0.00}{#1}}
\newcommand{\DecValTok}[1]{\textcolor[rgb]{0.68,0.00,0.00}{#1}}
\newcommand{\DocumentationTok}[1]{\textcolor[rgb]{0.37,0.37,0.37}{\textit{#1}}}
\newcommand{\ErrorTok}[1]{\textcolor[rgb]{0.68,0.00,0.00}{#1}}
\newcommand{\ExtensionTok}[1]{\textcolor[rgb]{0.00,0.48,0.65}{#1}}
\newcommand{\FloatTok}[1]{\textcolor[rgb]{0.68,0.00,0.00}{#1}}
\newcommand{\FunctionTok}[1]{\textcolor[rgb]{0.28,0.35,0.67}{#1}}
\newcommand{\ImportTok}[1]{\textcolor[rgb]{0.00,0.48,0.65}{#1}}
\newcommand{\InformationTok}[1]{\textcolor[rgb]{0.37,0.37,0.37}{#1}}
\newcommand{\KeywordTok}[1]{\textcolor[rgb]{0.00,0.48,0.65}{#1}}
\newcommand{\NormalTok}[1]{\textcolor[rgb]{0.00,0.48,0.65}{#1}}
\newcommand{\OperatorTok}[1]{\textcolor[rgb]{0.37,0.37,0.37}{#1}}
\newcommand{\OtherTok}[1]{\textcolor[rgb]{0.00,0.48,0.65}{#1}}
\newcommand{\PreprocessorTok}[1]{\textcolor[rgb]{0.68,0.00,0.00}{#1}}
\newcommand{\RegionMarkerTok}[1]{\textcolor[rgb]{0.00,0.48,0.65}{#1}}
\newcommand{\SpecialCharTok}[1]{\textcolor[rgb]{0.37,0.37,0.37}{#1}}
\newcommand{\SpecialStringTok}[1]{\textcolor[rgb]{0.13,0.47,0.30}{#1}}
\newcommand{\StringTok}[1]{\textcolor[rgb]{0.13,0.47,0.30}{#1}}
\newcommand{\VariableTok}[1]{\textcolor[rgb]{0.07,0.07,0.07}{#1}}
\newcommand{\VerbatimStringTok}[1]{\textcolor[rgb]{0.13,0.47,0.30}{#1}}
\newcommand{\WarningTok}[1]{\textcolor[rgb]{0.37,0.37,0.37}{\textit{#1}}}
\usepackage{longtable,booktabs,array}
\usepackage{calc} % for calculating minipage widths
% Correct order of tables after \paragraph or \subparagraph
\usepackage{etoolbox}
\makeatletter
\patchcmd\longtable{\par}{\if@noskipsec\mbox{}\fi\par}{}{}
\makeatother
% Allow footnotes in longtable head/foot
\IfFileExists{footnotehyper.sty}{\usepackage{footnotehyper}}{\usepackage{footnote}}
\makesavenoteenv{longtable}
\usepackage{graphicx}
\makeatletter
\def\maxwidth{\ifdim\Gin@nat@width>\linewidth\linewidth\else\Gin@nat@width\fi}
\def\maxheight{\ifdim\Gin@nat@height>\textheight\textheight\else\Gin@nat@height\fi}
\makeatother
% Scale images if necessary, so that they will not overflow the page
% margins by default, and it is still possible to overwrite the defaults
% using explicit options in \includegraphics[width, height, ...]{}
\setkeys{Gin}{width=\maxwidth,height=\maxheight,keepaspectratio}
% Set default figure placement to htbp
\makeatletter
\def\fps@figure{htbp}
\makeatother
\setlength{\emergencystretch}{3em} % prevent overfull lines
\providecommand{\tightlist}{%
  \setlength{\itemsep}{0pt}\setlength{\parskip}{0pt}}
\setcounter{secnumdepth}{5}
\makeatletter
\@ifpackageloaded{tcolorbox}{}{\usepackage{tcolorbox}}
\@ifpackageloaded{fontawesome}{}{\usepackage{fontawesome}}
\definecolor{quarto-callout-color}{HTML}{acacac}
\definecolor{quarto-callout-note-color}{HTML}{4582ec}
\definecolor{quarto-callout-important-color}{HTML}{d9534f}
\definecolor{quarto-callout-warning-color}{HTML}{f0ad4e}
\definecolor{quarto-callout-tip-color}{HTML}{02b875}
\definecolor{quarto-callout-caution-color}{HTML}{fd7e14}
\makeatother
\makeatletter
\makeatother
\makeatletter
\@ifpackageloaded{caption}{}{\usepackage{caption}}
\AtBeginDocument{%
\renewcommand*\figurename{図}
\renewcommand*\tablename{表}
}
\AtBeginDocument{%
\renewcommand*\listfigurename{List of Figures}
\renewcommand*\listtablename{List of Tables}
}
\@ifpackageloaded{float}{}{\usepackage{float}}
\floatstyle{ruled}
\@ifundefined{c@chapter}{\newfloat{codelisting}{h}{lop}}{\newfloat{codelisting}{h}{lop}[chapter]}
\floatname{codelisting}{Listing}
\newcommand*\listoflistings{\listof{codelisting}{List of Listings}}
\makeatother
\makeatletter
\@ifpackageloaded{caption}{}{\usepackage{caption}}
\@ifpackageloaded{subfig}{}{\usepackage{subfig}}
\makeatother
\ifLuaTeX
  \usepackage{selnolig}  % disable illegal ligatures
\fi
\usepackage[style=authoryear]{biblatex}
\addbibresource{../quarto.bib}

\title{Quarto 使用例 (Jupyter + Julia 1.6.3)}
\author{ill-identified}
\date{2021/10/2}

\begin{document}
\maketitle

{
\hypersetup{linkcolor=}
\setcounter{tocdepth}{3}
\tableofcontents
}
\begin{tcolorbox}[bottomtitle=1mm, toptitle=1mm, titlerule=0mm, left=2mm, title=\textcolor{quarto-callout-caution-color}{\faFire}\hspace{0.5em}注意, rightrule=.15mm, arc=.35mm, colback=white, toprule=.15mm, bottomrule=.15mm, colbacktitle=quarto-callout-caution-color!10!white, leftrule=.75mm, colframe=quarto-callout-caution-color, coltitle=black]
Quarto
は最近公開されたばかりで開発中なので\textbf{ここの記述もすぐ時代遅れになる可能性がある}ことに注意してほしい.
\end{tcolorbox}

Jupyter 上で Julia を動かすには
\href{https://github.com/JuliaLang/IJulia.jl}{IJulia.jl} が必要.

Markdown の確認

\begin{enumerate}
\def\labelenumi{\arabic{enumi}.}
\tightlist
\item
  番号付きの
\item
  箇条書き

  \begin{enumerate}
  \def\labelenumii{\arabic{enumii}.}
  \tightlist
  \item
    ネストも
  \item
    できる
  \end{enumerate}
\end{enumerate}

\hypertarget{ux6570ux5f0fux8868ux793a}{%
\section{数式表示}\label{ux6570ux5f0fux8868ux793a}}

ブラック=ショールズ方程式 (式~\ref{eq-black-scholes})

\begin{equation}\protect\hypertarget{eq-black-scholes}{}{
\frac{\partial \mathrm C}{ \partial \mathrm t } + \frac{1}{2}\sigma^{2} \mathrm S^{2}
\frac{\partial^{2} \mathrm C}{\partial \mathrm C^2}
  + \mathrm r \mathrm S \frac{\partial \mathrm C}{\partial \mathrm S}\ =
  \mathrm r \mathrm C 
}\label{eq-black-scholes}\end{equation}

テキスト出力テスト

\begin{Shaded}
\begin{Highlighting}[]
\StringTok{"Julia version: "} \OperatorTok{*} \FunctionTok{string}\NormalTok{(}\ConstantTok{VERSION}\NormalTok{)}
\end{Highlighting}
\end{Shaded}

\begin{verbatim}
"Julia version: 1.6.3"
\end{verbatim}

\begin{Shaded}
\begin{Highlighting}[]
\FloatTok{1} \OperatorTok{+} \FloatTok{1}
\end{Highlighting}
\end{Shaded}

\begin{verbatim}
2
\end{verbatim}

図~\ref{fig-p1} を見よ.

\begin{Shaded}
\begin{Highlighting}[]
\KeywordTok{using}\NormalTok{ Plots}
\NormalTok{x }\OperatorTok{=} \FloatTok{1}\OperatorTok{:}\FloatTok{10}\OperatorTok{;}\NormalTok{ y }\OperatorTok{=} \FunctionTok{rand}\NormalTok{(}\FloatTok{10}\NormalTok{)}\OperatorTok{;}
\NormalTok{Plots.}\FunctionTok{plot}\NormalTok{(x}\OperatorTok{,}\NormalTok{ y}\OperatorTok{,}\NormalTok{ fmt }\OperatorTok{=} \OperatorTok{:}\NormalTok{png)}
\end{Highlighting}
\end{Shaded}

\begin{figure}

{\centering \includegraphics{Jupyter-Julia_files/figure-pdf/fig-p1-output-1.png}

}

\caption{\label{fig-p1}テスト}

\end{figure}

図~\ref{fig-gadfly} を見よ.

\begin{Shaded}
\begin{Highlighting}[]
\KeywordTok{using}\NormalTok{ Gadfly}\OperatorTok{,}\NormalTok{ RDatasets}\OperatorTok{,}\NormalTok{ Compose}\OperatorTok{,} \BuiltInTok{Random}
\BuiltInTok{Random}\NormalTok{.}\FunctionTok{seed!}\NormalTok{(}\FloatTok{123}\NormalTok{)}

\FunctionTok{set\_default\_plot\_format}\NormalTok{(}\OperatorTok{:}\NormalTok{png)  }\CommentTok{\# }\AlertTok{TODO}
\FunctionTok{set\_default\_plot\_size}\NormalTok{(}\FloatTok{21}\NormalTok{cm}\OperatorTok{,} \FloatTok{8}\NormalTok{cm)}

\NormalTok{p1 }\OperatorTok{=}\NormalTok{ Gadfly.}\FunctionTok{plot}\NormalTok{(}\FunctionTok{dataset}\NormalTok{(}\StringTok{"ggplot2"}\OperatorTok{,} \StringTok{"mpg"}\NormalTok{)}\OperatorTok{,}
\NormalTok{     x}\OperatorTok{=}\StringTok{"Cty"}\OperatorTok{,}\NormalTok{ y}\OperatorTok{=}\StringTok{"Hwy"}\OperatorTok{,}\NormalTok{ label}\OperatorTok{=}\StringTok{"Model"}\OperatorTok{,}\NormalTok{ Geom.point}\OperatorTok{,}\NormalTok{ Geom.label}\OperatorTok{,}
\NormalTok{     intercept}\OperatorTok{=}\NormalTok{[}\FloatTok{0}\NormalTok{]}\OperatorTok{,}\NormalTok{ slope}\OperatorTok{=}\NormalTok{[}\FloatTok{1}\NormalTok{]}\OperatorTok{,}\NormalTok{ Geom.}\FunctionTok{abline}\NormalTok{(color}\OperatorTok{=}\StringTok{"red"}\OperatorTok{,}\NormalTok{ style}\OperatorTok{=:}\NormalTok{dash)}\OperatorTok{,}
\NormalTok{     Guide.}\FunctionTok{annotation}\NormalTok{(}\FunctionTok{compose}\NormalTok{(}\FunctionTok{context}\NormalTok{()}\OperatorTok{,}\NormalTok{ Gadfly.}\FunctionTok{text}\NormalTok{(}\FloatTok{6}\OperatorTok{,}\FloatTok{4}\OperatorTok{,} \StringTok{"ほげほげ"}\OperatorTok{,}\NormalTok{ hleft}\OperatorTok{,}\NormalTok{ vtop)}\OperatorTok{,} \FunctionTok{fill}\NormalTok{(}\StringTok{"red"}\NormalTok{))))}

\NormalTok{x }\OperatorTok{=}\NormalTok{ [}\FloatTok{20}\FunctionTok{*rand}\NormalTok{(}\FloatTok{20}\NormalTok{)}\OperatorTok{;} \FunctionTok{exp}\NormalTok{(}\OperatorTok{{-}}\FloatTok{3}\NormalTok{)]}
\NormalTok{D }\OperatorTok{=} \FunctionTok{DataFrame}\NormalTok{(x}\OperatorTok{=}\NormalTok{x}\OperatorTok{,}\NormalTok{ y}\OperatorTok{=} \FunctionTok{exp}\NormalTok{.(}\OperatorTok{{-}}\FloatTok{0.5}\FunctionTok{*asinh}\NormalTok{.(x)}\OperatorTok{.+}\FloatTok{5}\NormalTok{) }\OperatorTok{.+} \FloatTok{2}\FunctionTok{*randn}\NormalTok{(}\FunctionTok{length}\NormalTok{(x)))}
\NormalTok{abline }\OperatorTok{=}\NormalTok{ Geom.}\FunctionTok{abline}\NormalTok{(color}\OperatorTok{=}\StringTok{"red"}\OperatorTok{,}\NormalTok{ style}\OperatorTok{=:}\NormalTok{dash)}
\NormalTok{p2 }\OperatorTok{=}\NormalTok{ Gadfly.}\FunctionTok{plot}\NormalTok{(D}\OperatorTok{,}\NormalTok{ x}\OperatorTok{=:}\NormalTok{x}\OperatorTok{,}\NormalTok{ y}\OperatorTok{=:}\NormalTok{y}\OperatorTok{,}\NormalTok{  Geom.point}\OperatorTok{,}\NormalTok{  Scale.x\_asinh}\OperatorTok{,}\NormalTok{ Scale.y\_log}\OperatorTok{,}
\NormalTok{     intercept}\OperatorTok{=}\NormalTok{[}\FloatTok{148}\NormalTok{]}\OperatorTok{,}\NormalTok{ slope}\OperatorTok{=}\NormalTok{[}\OperatorTok{{-}}\FloatTok{0.5}\NormalTok{]}\OperatorTok{,}\NormalTok{ abline)}
\FunctionTok{hstack}\NormalTok{(p1}\OperatorTok{,}\NormalTok{ p2)}
\end{Highlighting}
\end{Shaded}

\begin{figure}

{\centering \includegraphics{Jupyter-Julia_files/figure-pdf/fig-gadfly-output-1.svg}

}

\caption{\label{fig-gadfly}Gadfly で作成したグラフ}

\end{figure}

文献引用

\autocite{R-quarto}, \textcite{R-rmdja}

\begin{Shaded}
\begin{Highlighting}[]
\FunctionTok{run}\NormalTok{(}\SpecialStringTok{\textasciigrave{}quarto render julia{-}test.ipynb\textasciigrave{}}\NormalTok{)}
\end{Highlighting}
\end{Shaded}

\begin{Shaded}
\begin{Highlighting}[]
\FunctionTok{run}\NormalTok{(}\SpecialStringTok{\textasciigrave{}quarto render julia{-}test.ipynb {-}{-}to pdf\textasciigrave{}}\NormalTok{)}
\end{Highlighting}
\end{Shaded}


\printbibliography

\end{document}
